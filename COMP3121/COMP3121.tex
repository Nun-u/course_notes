\documentclass{article}

% Include necessary packages
\usepackage{amsmath} % for mathematical notation
\usepackage{amssymb} % for mathematical symbols
\usepackage{algorithm} % for typesetting algorithms
\usepackage{algorithmic} % for typesetting algorithms
\usepackage{listings} % for typesetting code
\usepackage{color} % for syntax highlighting
\usepackage{geometry} % for adjusting margins

\geometry{margin=1in}

\title{COMP3121 Notes}
\begin{document}

\section*{\huge COMP3121 Notes}
~\\
\section{Week 1}
\subsection{Intro to Algorithms}

\textbf{What is an Algorithm?} \\
A collection of precisely defined steps that can be executed mechanically (without intelligent decision-making).
\\\\
\textbf{Sequential Deterministic Algorithms:} \\
Algorithms are given as sequences of steps, thus assuming that only one step can be executed at a given time.
\\\\
\textbf{Example: Two Thieves} \\
Alice and Bob have robbed a warehouse and have to split a pile of items without price tags on them. 
Design an algorithm to split the pile so that each thief \textbf{believes} that they have got at least
half the loop.
\\\\
\underline{Solution:} \\
\begin{algorithmic}
    \STATE{Alice splits the pile in two parts, so that she believes that both parts are equal}
    \STATE{Bob then picks the part that he believes is no worse than the other}
\end{algorithmic}
~\\
\textbf{Example: Three Thieves} \\
Alice, Bob and Carol have robbed a warehouse and have to split a pile of items without price tags on them. How do they do this in a way that ensures that each thief \textbf{believes} they have gotten at least one third of the loot.
\\\\
\underline{Solution:} \\
\begin{algorithmic} 
    \STATE{Alice makes a pile of \(\frac{1}{3}\) called \(X\)}
    \IF{Bob agrees \(X\) \(\le \frac{1}{3}\)}
        \STATE{Bob agrees to split the remainder with Carol}
        \IF{Carol agrees \(X\) \(\le \frac{1}{3}\)}
            \STATE{Bob and Carol split the rest}
        \ELSE{Alice and Bob split the rest}
        \ENDIF
    \ELSE{Bob reduces pile until he thinks \(X\) \(\le \frac{1}{3}\) and Alice and Carol split the rest}
    \ENDIF
\end{algorithmic}
~\\
\textbf{When are proofs necessary?} \\
We use proofs in circumstances where it is not clear that an algorithm truly does its job. \\\\
Proofs should \underline{not be used to prove the obvious.}
\end{document}
